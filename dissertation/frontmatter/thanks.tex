%!TEX root = ../dissertation.tex
% the acknowledgments section

\doublespacing

The mathematician Maryam Mirzakhani described working on a new proof as ``like being lost in a jungle and trying to use all the knowledge that you can gather to come up with some new tricks.'' 
``With some luck,'' she said, ``you might find a way out.''
The same could be said of any time spent at the frontiers of understanding.
The predominant mental state is confusion, narrowly edging out paranoia of errors and compulsion for answers.
Writing a dissertation requires spending a great deal of time marinating in this noxious brew.
Even accounting for luck, it may not be reasonable to expect to ever find one's way out.
But thanks to the generous support and guidance of colleagues, friends, and family, the longer I've spent in the jungle, the more I've come to appreciate its beauty.

No one channels that sense of beauty better than my advisor Noah Goodman, who has been a constant source of intellectual inspiration.
Half the trouble is picking the right jungle to go thwacking into, and Noah is a brilliant curator of problems.
I've left every conversation brimming with more ideas than I could ever seriously act on.
Over five years of near-constant interaction, it's been a joy to not only coordinate on ways of talking about complex ideas but also to find my patterns of thinking slowly adapting as well.
I'm grateful to now have a functioning simulator of Noah's voice installed in the back of my mind, encouraging me to forge new paths forward.

Intellectual life at a university revolves around many overlapping communities, each of which nurtures a distinct culture and mode of thought.
The center of all of these communities has been the CoCoLab, where I've had the fortune to overlap with a lively and brilliant cast of characters: past, present, or just passing through. 
Thanks to Long Ouyang, Desmond Ong, Andreas Stuhlm\"uller, Sid Narayanaswamy, Dan Hawthorne, Daniel Ly, Justine Kao, Greg Scontras, Leon Bergen, Judith Degen, Erin Bennett, MH Tessler, Ben Peloquin, Jason Freeman, Katherine Hermann, JP Chen, Sahil Chopra, Ishita Dasgupta, Bill McDowell, Mike Wu, Bal\'azs T\"or\"ok, Meg Sano, and Judy Fan.
Among countless memories, I'm especially thankful for the wisdom generously offered by Leon and Judith, and for lessons on the finer points of pour overs amidst long conversations with MH. 

Our department raises few boundaries between different research groups, and I've found warm and welcoming communities and mentors in the Social Learning lab, the Language and Cognition lab, and the NeuroAI Lab. 
Hyo Gweon's perspicacious questions and insights have consistently deepened my understanding of social cognition and theory of mind. 
Mike Frank has been a true role model for how to navigate the endless demands of academia; teaching, writing, and working with you taught me how to be a more responsible, transparent, and clear-thinking scientist.
Finally, meeting Dan Yamins was a truly pivotal moment for my intellectual development.
Our expansive conversations pushed me to articulate more clearly the problems I truly cared about and the kind of scientist I want to be.

%Other not-entirely-academic communities provided the support and outlet for other enthusiasms to keep me sane.
%The climbing community at Stanford , even though I hurt my finger and stopped coming to the gym.
%Every week at Thallone, I looked forward to most thoughtful groups of people.

My time at Stanford was punctuated by several transformative academic visits to Europe, which allowed me to step back from the frantic pace of Silicon Valley and take stock of the broader landscape.
Key portions of this dissertation were conceived through the hospitality and energy of Michael Franke at T\"ubingen, Kenny Smith at Edinburgh, and Riccardo Fusaroli at Aarhus.

Thank you to Herb Clark for being not only the giant whose shoulders my work stands upon but also a kind and generous mentor.
It comforts to me to know that whenever I think I've figured out one small piece of the puzzle, your body of work is waiting with the next challenge. 

Thank you to Judy Fan, who has shaped my thinking more than anyone else. You are my closest partner, collaborator, and counselor, and your influence is evident in everything I do.

Lastly, thank you to my family. My brother Jonathan has been a steadfast companion, and my parents Jan and Thomas have infused my life with the unconditional love and support to grow intellectually and spiritually. You've always been in my corner, and I wouldn't be here without you.
