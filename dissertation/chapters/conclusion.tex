%!TEX root = ../dissertation.tex
\chapter{Conclusion}
\label{conclusion}


Language is not some monolithic body of knowledge that we acquire at an early age and deploy mechanically for the rest of our lives. Nor is its evolution a slow, inter-generational drift. It is a means for communication -- a shared interface between minds -- and must therefore adapt over the rapid timescales required by communication. In other words, we are constantly learning language. Not just one language, but an enormous family of related languages, across every repeated interaction with every partner. 
We conclude by discussing some broader questions raised by the theoretical perspective we have advanced here.

\section{Generalization in language acquisition}

Throughout our work, we have assumed a discourse-level structure to an agent's priors. 
We assume there is uncertainty over how words are used \emph{in the given conversation, by the current partner}. 
However, there is a broader debate over the timescales at which lexicons and lexicon learning mechanisms operate.
In particular, these hierarchical learning mechanisms suggest the possibility of a developmental parallel.
Are the lexical learning mechanisms adults use to coordinate on local conventions \emph{within} an interaction the same as those supporting language-learning more broadly? 

Most laboratory tasks investigating cross-situational word learning only use a single speaker, and even sophisticated models of cross-situational word learning that account for pragmatic reasoning about speaker intentions \cite[e.g.]{FrankGoodmanTenenbaum09_Wurwur} tend to collapse over \emph{who} is talking. 
Yet, as we have argued throughout this work, there is substantial variability across different speakers. 
If the majority of child-directed speech only comes from a single primary caregiver, then the child may face a difficult generalization problem once they begin interacting with others. 
Upon hearing an unfamiliar word from a novel speaker, or a familiar word utterance with an unfamiliar meaning, it could be a quirk of that particular speaker \emph{or} indicative of a globally shared convention. 
There may therefore be substantial path-dependence in acquisition, as children develop their lexical prior and become better attuned to the overall variability in the population \cite<see>[Chap. 6]{Clark09_FirstLanguageAcquisition}. 

This slow-developing lexical prior is one of several explanation for why young children are so terrible at coordinating on local conventions in repeated reference games \cite{GlucksbergKraussWeisberg66_DevoRefGames,KraussGlucksberg77_SocialNonsocialSpeech}. 
When an experimenter feeds them the messages that adult speakers produced naturally, they had no trouble, even as they reduced down to one- or two-word utterances. When they played with one another, however, 
Kindergardeners continued to make errors even after 15-16 repetitions; children as old as fifth grade only improved with assistance from the experimenter and never approached the perfect levels of adult performance. 
Instead of beginning with the long indefinite descriptions full of hedges and modifiers that adults provide, nursery-school speakers began with short, highly idiosyncratic descriptions like \emph{Mother's dress}. 
If adult speakers' long hedge-filled messages are indeed motivated by lexical uncertainty, then perhaps young children have simply not obtained enough linguistic variability to calibrate their lexical prior. 
Alternatively, if the pragmatic reasoning required to produce informative utterance depends on theory of mind, then the high processing demands of the task may simply be inhibited performance \cite<e.g.>{SetohScottBaillargeon16_FalseBelief}. This remains an under-explored puzzle for future developmental research. 

\section{Mechanisms for adaptation}

While we have focused on how participants coordinate on \emph{lexical} meaning, this is only one of many levels at which conventions may form. 
In more complex circumstances, there is often initial uncertainty not just about which of a small set of targets a particular message refers to, but how to represent the relevant targets of reference in the first place. 
Learning to communicate effectively may require discovering a lower-dimensional representation in which the targets of reference vary.
For instance, when using sketches to communicate about the identity of complex pieces of music \cite{HealeySwobodaUmataKing07_GraphicalLanguageGames}, a particular set of strokes could correspond to any number of properties (pitch, tempo, melody, rhythm, intensity) at any temporal granularity. 
This is made particularly clear in a classic maze game \cite{GarrodAnderson87_SayingWhatYouMean}: in order to give effective spatial directions, speakers had well-tuned lexical priors but had to coordinate on what space of \emph{referents} to use (e.g. paths, coordinates, lines, landmarks). 

Prior theories have assumed representations of lexical meanings are relatively fixed and the only learning taking place is how one's partner construes a multi-stable percept. 
For examine, this seems to be what \citeA{BrennanClark96_ConceptualPactsConversation} had in mind when they coined the term \emph{conceptual pact}, and \cite{stolk2016conceptual} have influentially argued that partners in communication construct shared conceptual spaces. 
Given present data it is not clear how these two sources of uncertainty could be teased apart, though certain kinds of conventions (e.g. proper names or acronyms) seem to rely more on binding new linguistic tokens to meanings than on constructing new conceptualizations.
Thus, we expect both levels of coordination are likely to play an important role. 
Our probabilistic model could be extended to handle additional levels of coordination by placing uncertainty over a hyper-parameter corresponding to the intended feature dimension that must be jointly inferred with the correspondence along that dimension. 

\section{Conventions on networks}

In addition to updating our model of a particular partner based on immediate feedback, i.e. utterances and choices made in previous rounds of the game, a Bayesian learning model predicts that sparse observations of a partner's language use may license much broader inferences about their lexicon via diagnostic information about their social group or background. If someone's favorite song is an obscure B-side from an obscure Boston hardcore band, you can make fairly strong inferences about what else they like to listen to and how similar they might be to you \cite{VelezEtAl16_Overlaps, GershmanEtAl17_StructureSocialInfluence}. Similarly, if someone casually refers to an obscure New York landmark you also recognize, you can safely update your beliefs about their lexicon to include a number of other conventions shared among New Yorkers. Lexica cluster within social groups, so inverting this relationship can yield rapid lexical learning from inferences about social group membership.

This source of lexical learning was explored in a study by \citeA{IsaacsClark87_ReferencesExpertsNovices} where novices and experts were paired for a repeated reference game using postcards of New York landmarks. Both directors and matchers could be either novices or experts, creating a 2x2 design. While a strong main effect of reduction was found across all pairings of experts and novices, they differed strikingly in their use of proper nouns (i.e. conventions shared by experts). For instance, over the course of the experiment, experts consistently used short messages with proper nouns (e.g. ``the Rockefeller Center'') when talking to other experts, while novice directors gradually adapted to expert matchers, doubling their use of proper names (and therefore drastically reducing the length of their utterances).

Most striking, however, was the observation that directors had already adapted in the first few trials of the first round: by the fourth round expert directors were already using a proper noun three times more often when talking to other directors than in talking to novices. In fact, independent raters were presented with transcripts from the first two postcards and correctly judged the expertise of the two partners 84\% of the time. This is a straightforward prediction of a hierarchical Bayesian model like the one we proposed in Chapter 2: given a latent group representation of New Yorkers, a director can make a strong prediction that if their partner belongs to this group, ``Rockefeller Center'' will belong to their lexicon with high probability. Hence, any interpretation failure is strong evidence that their partner is not in the group and is thus equally unlikely to recognize ``Citicorp Building'' or ``Brooklyn Bridge''. In this way, convention formation and social group inference are intimately intertwined. 


